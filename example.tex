\documentclass[english]{article}%

% Import the package:
\usepackage{notation}

% Optional: add this if you want the table of symbols
\makenomenclature

% Basic syntax: 
%   1. Optional description used to generate table of notation 
%      (if empty, this notation will not be included in the table of symbols,
%      but will still be hyperlinked)
%   2. Name of the notation macro (without the backslash)
%   3. Value of the notation
\notation[A small difference]{smallDiff}{\delta}

% More advanced: if you want to override existing symbols, use a star (notation*),
% and save the old value of the symbol to avoid infinite recursion during compilation
\let\rawpi\pi
\notation*[An important constant]{pi}{\rawpi}

% Defining abbreviations:
\abbreviation[A hidden Markov model]{HMM}{hidden Markov model}

\begin{document}

\begin{enumerate}
  \item Here is my first macro: $\smallDiff$.
  \item Here is another one: \[\pi\].
  \item Here is an abbreviation: \HMM.
\end{enumerate}

\newpage

\begin{enumerate}
  \item Here is my macro again, which is a link to the first occurrence above now: $\smallDiff$.
  \item Another link: $\pi$.
  \item Here is my abbreviation again: a \HMM, several \HMMs.
\end{enumerate}

% Pick a name for your list of symbols:
\renewcommand{\nomname}{My list of Symbols}

% Place the list of symbols here
\printnomenclature 

\end{document}